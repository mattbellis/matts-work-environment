%%%%%%%%%%%%%%%%%%%%%%%%%%%%%%%%%%%%%%%%%%%%%%%%%%%%%%%%%%%%%%%%%%%%%%%%%%%%%%%%
% From here
% http://tex.stackexchange.com/questions/47924/creating-playing-cards-using-tikz
%%%%%%%%%%%%%%%%%%%%%%%%%%%%%%%%%%%%%%%%%%%%%%%%%%%%%%%%%%%%%%%%%%%%%%%%%%%%%%%%

\documentclass{minimal}
%\usepackage[a4paper,margin=1cm]{geometry}
\usepackage[letterpaper,margin=1cm]{geometry}
\usepackage{tikz}
\usepackage{multicol}
\usepackage{amsthm,amsmath,amssymb}
\usepackage[protrusion=true,expansion=true]{microtype}
\usetikzlibrary{positioning,shapes,shadows,arrows,backgrounds,fit}
    \usepackage[utf8]{inputenc}
    \begin{document}
    \tikzstyle{textstyle}=[rectangle, text width=3.5cm, text badly ragged, scale=0.8]
    \begin{multicols}{3}
    \begin{center}
        \begin{tikzpicture}[background rectangle/.style = {draw=black, fill=white,
            rounded corners}, show background rectangle, node distance=0.3cm]
            \node (side) [textstyle, fill=red, rotate=90, text width=4cm, scale=1.4, text centered] {%
                \begin{tabular}{r}
                TIME
                    \end{tabular}
            };
        \node (kind) [textstyle, right=of side, scale=1.5] {\textbf{KEEPER}};
        \node (desckind) [textstyle, below=of kind] {
            When you play this card, bla bla bla.
        };
        \node (title) [textstyle, below=of desckind, scale=1.3] {Time};
        \node (separator) [thick, fill=black, below=of title, text width=3.5cm] {};
        \node (description) [textstyle, below=of separator] {
            The player bla bla bla bla
                \[ H \Psi = \nabla \Psi \]
                \[ J \Phi = \nabla \Phi \]
                and then bla bla bla.
        };
    \end{tikzpicture}
    \end{center}
\end{multicols}
\end{document}
