% http://stefan.endrullis.de/en/game_set.html
%%%%%%%%%%%%%%%%%%%%%%%%%%%%%%%%
%    set.tex - version 1.0     %
%  by Stefan Endrullis (2004)  %
%%%%%%%%%%%%%%%%%%%%%%%%%%%%%%%%

% die Wurzel aus der Anzahl der Karten pro Seite
\def\cardsPerPageSqrt{2}
% gibt an, ob der Rahmen der Karten auch an den Blatträndern gezeichnet werden soll
\def\pageFrame{1} % 1 für Rahmen, 0 wenn kein Rahmen gezeichnet werden soll

\documentclass[10pt,fleqn,a4paper]{article}

% Packages einbinden
\usepackage{ifthen}
\usepackage[nomessages]{fp}
\usepackage{pstricks, pst-node, pst-tree, pst-text}

% gibt das Ergebnis der Division 1/x zurück für x \in {1, 2, 3, 4, 5}
\newcommand{\einsDurch}[1]{%
    \FPeval{\einsDurchTmp}{1/#1}% Berechnung des Wertes
    \einsDurchTmp% Ausgabe des Wertes
}%

% gibt die aktuelle Farbe des Rahmens der Formen zurück
\newcommand{\setFrameColor}{%
    \ifthenelse{\theframeColorNo = 0}{\psset{linecolor=red}}{}%
    \ifthenelse{\theframeColorNo = 1}{\psset{linecolor=green}}{}%
    \ifthenelse{\theframeColorNo = 2}{\psset{linecolor=blue}}{}%
}%

% gibt die aktuelle Farbe des Rahmens der Formen zurück
\newcommand{\setFillColor}{%
    \ifthenelse{\thefillColorNo = 0}{\psset{fillcolor=white}}{}%
    \ifthenelse{\thefillColorNo = 1}{%
        \ifthenelse{\theframeColorNo = 0}{\psset{fillcolor=lightred}}{}%
        \ifthenelse{\theframeColorNo = 1}{\psset{fillcolor=lightgreen}}{}%
        \ifthenelse{\theframeColorNo = 2}{\psset{fillcolor=lightblue}}{}%
    }{}%
    \ifthenelse{\thefillColorNo = 2}{%
        \ifthenelse{\theframeColorNo = 0}{\psset{fillcolor=red}}{}%
        \ifthenelse{\theframeColorNo = 1}{\psset{fillcolor=green}}{}%
        \ifthenelse{\theframeColorNo = 2}{\psset{fillcolor=blue}}{}%
    }{}%
}%

% zeichnet eine Karte
\newcounter{drawCardI}
\newcommand{\drawCard}{%
    % die 1 bis 3 Formen zeichnen
    \setcounter{drawCardI}{0}%
    \whiledo{\thedrawCardI < \theformCount}{%
        % Position der Form berechnen
        \FPeval{\firstx}{\cardx+\formLeftMargin}% berechnet die X-Koordinate des ersten Punktes der Form
        \FPeval{\firsty}{\cardy+\formTopMargin+((3-\theformCount)*\formHalfHeightPlusSpace)+(\thedrawCardI*(\formHeight+\formVerticalSpace))}% berechnet die Y-Koordinate des ersten Punktes der Form
        % Farben setzen
        \setFrameColor\setFillColor%
        % Form zeichnen
        \rput[bl](\firstx,\firsty){\drawForm}%
        \stepcounter{drawCardI}%
    }%
}%

% zeichnet eine Form
\newcommand{\drawForm}{%
    \ifthenelse{\theformNo = 0}{\psframe(0,0)(\formWidth,\formHeight)}{}%
    \ifthenelse{\theformNo = 1}{\pstriangle(\formHalfWidth,\formHeight)(\formWidth,-\formHeight)}{}%
    \ifthenelse{\theformNo = 2}{\psellipse(\formHalfWidth,\formHalfHeight)(\formHalfWidth,\formHalfHeightMinusSpace)}{}%
}%

% zeichnet die Abgrenzungen zwischen den Karten und zum Rand des Blattes
\newcounter{drawFramesI}% eine lokale Zählvariable
\newcounter{drawFramesEnd}%
\newcommand{\drawFrames}{%
    \ifthenelse{\pageFrame = 1}{\setcounter{drawFramesI}{0}}{\setcounter{drawFramesI}{1}}% Startwert setzen
    \setcounter{drawFramesEnd}{\cardsPerPageSqrt}%
    \ifthenelse{\pageFrame = 1}{\stepcounter{drawFramesEnd}}{}% Endwert setzen
    \FPeval{\drawFramesLength}{4*\cardsPerPageSqrt}% berechnet der Länge der Striche zur Abgrenzung der Karten
    % alle Positionen durchgehen und die Abgrenzungen (Rahmen der Karten) zeichnen
    \whiledo{\thedrawFramesI < \thedrawFramesEnd}{%
        % Position berechnen
        \FPeval{\drawFramesPos}{4*\thedrawFramesI}% berechnet der Länge der Striche zur Abgrenzung der Karten
        % horizontale und vertikale Abgrenzung an der bestimmten Position zeichnen
        \rput[bl](0,\drawFramesPos){\psline[linecolor=black, linewidth=0.1pt](0,0)(\drawFramesLength,0)}%
        \rput[bl](\drawFramesPos,0){\psline[linecolor=black, linewidth=0.1pt](0,0)(0,\drawFramesLength)}%
        % Zähler um 1 erhöhen
        \stepcounter{drawFramesI}%
    }%
}%

%----------------------------------------------------------------%

% linken und oberen Rand entfernen
\pagestyle{empty}
\topmargin-1.65in
\textheight\paperheight
\linewidth 30cm
\oddsidemargin-1in
\parindent0mm

% helle Farben definieren
\newrgbcolor{lightred}{1 0.8 0.8}
\newrgbcolor{lightgreen}{0.8 1 0.8}
\newrgbcolor{lightblue}{0.8 0.8 1}

% counter definieren
\newcounter{cardsOnPage}     % die Anzahl der Karten, die auf die Seite gezeichnet wurden, beginnend bei 0
\newcounter{frameColorNo}    % die Nummer der Farbe für den Rahmen (0 = rot, 1 = grün, 2 = blau)
\newcounter{fillColorNo}  % die Nummer der Farbe für den Inhalt (0 = rot, 1 = grün, 2 = blau)
\newcounter{formNo}            % die Nummer der From (0 = Viereck, 1 = Dreieck, 2 = Ellipse)
\newcounter{formCount}       % die Anzahl der From auf einer Karte, beginnend bei 0

% Höhe, Breite und Abstand von Formen definieren
\def\formHeight{0.8}
\def\formWidth{2.4}
\def\formVerticalSpace{0.2}

% Vorberechnungen für Karten durchführen
\FPeval{\formTopMargin}{(4-(3*\formHeight)-(2*\formVerticalSpace)) / 2}% berechnet den Abstand einer Form nach oben
\FPeval{\formLeftMargin}{(4-\formWidth)/2}% berechnet den Abstand einer Form nach links
\FPeval{\formHalfHeightPlusSpace}{(\formHeight+\formVerticalSpace)/2}% berechnet (Höhe+Platz_nach_unten)/2
\FPeval{\formHalfHeightMinusSpace}{(\formHeight-\formVerticalSpace)/2}% berechnet (Höhe+Platz_nach_unten)/2
\FPeval{\formHalfHeight}{\formHeight/2}% berechnet (Höhe+Platz_nach_unten)/2
\FPeval{\formHalfWidth}{\formWidth/2}% berechnet (Höhe+Platz_nach_unten)/2

\begin{document}

%----------------------------------------------------------------%

% Seite so einrichten, daß ein (0,0)(4,4)-Rechteck die gesamte Seite ausfüllt
\def\xunit{5.246}
\def\yunit{-7.424}
% nun das Koordinatensystem so skalieren, daß genau \cardsPerPageSqrt^2 (0,0)(4,4)-Rechtecke eine Seite ausfüllen
\FPeval{\xunit}{\xunit/\cardsPerPageSqrt}%
\FPeval{\yunit}{(\yunit)/\cardsPerPageSqrt}%
\FPeval{\linewidta}{6/\cardsPerPageSqrt}%
%\def\yunit{-\yunit}%  \FPeval{\drawFramesLength}{4*\cardsPerPageSqrt}% berechnet der Länge der Striche zur Abgrenzung der Karten

\psset{xunit=\xunit, yunit=\yunit, linewidth=\linewidta mm, framearc=0, fillstyle=solid}

% Kartenpos auf (0,0) setzen
\def\cardx{0} \def\cardy{0}

% init some counter
\newcounter{rowi}% die aktuelle Zeilennummer
\newcounter{coli}% die aktuelle Spaltennummer
\setcounter{rowi}{0}% die aktuelle Zeilennummer
\setcounter{coli}{0}% die aktuelle Spaltennummer

% alle Eigenschaftkombinationen der Karten in 4 Schleifen durchgehen
\setcounter{frameColorNo}{0}
\whiledo{\theframeColorNo < 3}{%
    %  \theframeColorNo
    \setcounter{fillColorNo}{0}%
    \whiledo{\thefillColorNo < 3}{%
        \setcounter{formNo}{0}%
        \whiledo{\theformNo < 3}{%
            \setcounter{formCount}{1}%
            \whiledo{\theformCount < 4}{%
                % Karte zeichnen
                \drawCard%
                %
                % nächste Spalte bzw. neue Zeile bzw. neue Seite
                \stepcounter{coli}%
                \FPeval{\cardx}{\cardx+4}%
                \ifthenelse{\thecoli = \cardsPerPageSqrt}{%
                    \setcounter{coli}{0}\stepcounter{rowi}%
                    \def\cardx{0}%
                    \FPeval{\cardy}{\cardy+4}%
                }{}%
                \ifthenelse{\therowi = \cardsPerPageSqrt}{%
                    \drawFrames%
                    \setcounter{rowi}{0}\newpage%
                    \def\cardy{0}%
                }{}%
                %
                \stepcounter{formCount}%

            }%
            \stepcounter{formNo}%
        }%
        \stepcounter{fillColorNo}%
    }%
    \stepcounter{frameColorNo}%
}%
\addtocounter{coli}{\therowi}%
\ifthenelse{\thecoli > 0}{\drawFrames}{}%

% Testen, ob ein 4*4Rechteck die gesamte Seite ausfüllt
%\FPeval{\iY}{\Ypsz * \iy / (\iy + 1.0)}  % Calculates Y-coordinate
%\FPeval{\iX}{(\Ypsz - \iY)* \ix / \Xpsz} % Calculates X-coordinate

%\FPeval{\abcb}{2.0 * \abcb}
%\psframe[linecolor=red, linewidth=2pt, framearc=0, fillstyle=solid, fillcolor=blue](\abca,\abca)(\abcb,\abcb)

%\setcounter{multi}{1}
%\whiledo{\themulti < \}{ \addtocounter{multa}{#1} \stepcounter{multi} }
%a\\ \\ \\
%\arabic{multi}
%\psframe[linecolor=red, framearc=0, fillcolor=lightred](\abc,0.6)(3.2,1.4)
%\psframe[linecolor=red, framearc=0, fillcolor=lightred](\abc,1.6)(3.2,2.4)
%\psframe[linecolor=red, framearc=0, fillcolor=lightred](\abc,2.6)(3.2,3.4)

%\pstriangle(2,.5)(4,2)
%\rput[bl](1,1){
%\psellipse[linecolor=blue, linewidth=2pt, framearc=4.3, fillstyle=solid, fillcolor=lightgreen](2,-2)(2,1)
%}

%\psframe[linecolor=red, linewidth=2pt, framearc=0, fillstyle=solid, fillcolor=blue](2,-4)(4,2)

%----------------------------------------------------------------%


\end{document}
